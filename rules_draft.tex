\documentclass[twocolumn]{amsart}
\usepackage[utf8]{inputenc}
\usepackage[T1]{fontenc}
\usepackage[english]{babel}

\usepackage{lmodern}
\usepackage[margin=0.7in]{geometry}
\usepackage{amsmath}
\usepackage{booktabs}
\usepackage{tabularx}
\usepackage[perpage]{footmisc}
\usepackage{microtype}
\usepackage{varioref}
\usepackage{cleveref}

\renewcommand*{\thefootnote}{\fnsymbol{footnote}}
\newcommand{\die}{\text{d}}

\title{Some rules}
\date{}

\begin{document}

\maketitle

Quick character creation:

\begin{enumerate}
\item Generate \textsc{diccs} scores, $1\die10$, rerolling 10s, divide by 2 rounding up
\item Choose a class
\item Make up skills (5 points, 2 more if Rogue)
\item Make up inventory
\item Make up a name, personality, and physical description
\end{enumerate}

The basics of play, real fast:

\begin{itemize}
\item If you want to do something difficult and you have net advantage, roll $X\die10$ (where $X$ is the number of distinct advantages you have) and take the highest
\item If you want to do something difficult and you have net disadvantage, roll $X\die10$ (where $X$ is the number of distinct disadvantages you have) and take the lowest
\item If you get hit, roll $C\die10$ (where $C$ is your Constitution) and subtract the lowest from your Health
\item If lots of people are trying to do stuff at once, declare actions in order of Dexterity, from lowest to highest; resolve actions from highest to lowest
\end{itemize}

\section{Characters}

You need to make a character to interact with the world.

Characters have attributes, classes, skills, and equipment.

The attributes are: Dexterity, Intelligence, Charisma, Constitution, and Strength.
You can remember them with the acronym \textsc{diccs}.
To generate your attribute scores, roll $1\die10$ for each attribute in the list.\footnotemark
If you get a 10, reroll until you have a different number.
Your attribute is the number you rolled divided by 2, rounding up.

\footnotetext{$X\die Y$ is shorthand for $X$ number of $Y$-sided dice.}

You can choose your class from the list: Fighter, Rogue, Magic-User.

You can choose your skills; just make up some things you're good at.
You get 5 skill points.
Spend 1 skill point for each skill you choose
These are your only skills.
If you choose fewer than 5 skills, you can raise a skill by 1 by spending 1 skill point.

You can make up your inventory as you please.
You get 5 items, plus clothing and a waterskin.

\section{Classes}

\subsection{Fighter}
Fighters can fight.
Unless the Fighter is surprised or faced with overwhelming odds, it is impossible for any human other than another Fighter to attack them with advantage.
Moreover they attack all humans except other Fighters with advantage.
If two Fighters fight, the Fighter with the higher level has advantage.

They are the only class able to wear heavy armor and wield two-handed weapons without being disadvantaged.

If a Fighter successfully hits something, they add their level to the damage rolled to determine total damage done.

Fighters are \emph{required} to have at least one combat-related skill.

\subsection{Rogue}
Rogues can move quickly.
They always move before every other human, and often before animals as well.

Rogues have 2 extra skill points at character creation.
Every time they level up, they gain 1 skill point.

\subsection{Magic-User}
Magic-Users can use magic.
They can only do this when they have their grimoire with them.
Note that this need not be a book---any way the formulas could be preserved and carried, works, from tattoos to inscribed chains.
At creation Magic-Users get 3 random spells.
See \vref{magic} for more.

\section{Skills}\label{skills}

We want to quantify skill differentials.
If a thing is hard to do but you are trained to do it, you have some number between 1 and 9 telling you how good you are at it.
Each skill should fall under some attribute.
Each skill has a value equal to that of the associated attribute, plus the number of skill points spent on that skill.

\section{Doing things}

In general if you want to do something just like say it and it'll be done.
Like come on you obviously know how to open doors and walk up stairs.
But not everything is that easy.

Say you want to do something hard.
Then you roll a die to see if you succeed.
If you're rolling against an inanimate obstacle, the \textsc{dm} will set a difficulty level that you have to roll over.
If you're rolling against an animate obstacle, you will roll against the \textsc{dm}, and whoever gets the highest wins.

\subsection{Better and worse odds}
But not all hard tasks are equally hard.
People can have advantages and disadvantages from all sorts of things---their abilities, their equipment, their fortitude, their dedication, and general circumstance.

Advantages and disadvantages are determined ad hoc.
The only set guideline for determining advantages is: for each of your Health points below zero, take a disadvantage.
Otherwise you take advantages for anything you can convince me will help you, and take disadvantages for anything I think will hinder you.

\paragraph{An important caveat}
Advantages and disadvantages only apply once in a given check.
You can't double-dip.
So if I'm fighting you and I'm standing above you on a hill, then I get an advantage for having the high ground, but you don't also get a disadvantage for having the low ground.

Count your advantages and disadvantages.
Subtract the number of disadvantages from the number of advantages.
If the number is positive, roll $1+\text{that number}$ dice; if negative, roll $1 - \text{that number}$ dice.
Take the highest number rolled if you have more advantages than disadvantages.
Take the lowest number rolled if you have more disadvantages than advantages.

\subsection{Multiple-user checks}
This will go clearest by example.
Say John and Marsha are wrestling while Bill is trying to drop a boulder on John.
To figure out what happens, we'd count up all the different actions and advantages and everybody involved would roll at once.
\emph{Only the action of the person who rolled highest will occur.}
So \emph{either} John will put Marsha in a headlock \emph{or} Marsha will throw John off \emph{or} Bill's boulder will hit John (and if this happens there's a decent chance Marsha will be hit as well!).

\subsection{Doing things at the same time}
It often happens that you want to do something at the same time as someone else, and their success is somehow tied up with yours.
A prime example is combat.
In such cases, every involved party will declare what they are they are trying to do in ascending order of Dexterity.
Lower-Dexterity people say their action first, so the higher-Dexterity people can hear them and plan accordingly.
If two people are tied and allied they should decide the order that they will declare their actions in themselves.
If they're not under the same banner the person with the highest level goes first.
If their levels are tied, flip a coin.

\section{Hitting things and getting hit}\label{weapon}

All weapons do the same amount of damage.
If you try and hit someone, make a check like in any other difficult action.
If you win you do damage.

\subsection{Health and damage}

Everybody has some Health.
The maximum amount of Health they can have is equal to their Constitution.
To see how much damage you do, roll dice equal to your target's Constitution and take the lowest number.
If you are a Fighter, add your level to that number.
Then subtract that number from their Health.
If they go below 0 they get knocked out---flip a coin each round to see if they recover.
Also, they're injured---see \vref{injuries}.
If they get medical attention they can recover Health but if their injury is the sort that can't be healed, like a severed limb, it won't be.
Without medical attention humans recover 1 Health per day.

\begin{table}
\caption{Injuries}
\label{injuries}
\begin{tabularx}{\linewidth}{@{}lX@{}}
\toprule
$1\die6 - H$\footnotemark & Injury \\
\midrule
2 & Cut or bruise. Will leave a scar or permanent bump. \\
3 & Weapon arm broken. Attack with disadvantage for a month. \\
4 & Face blow. Save\footnotemark\ or lose an eye. \\
5 & Crotch shot. Save or lose your bits. \\
6 & Limb wrecked. Lose $1\die6$: 1-2. a leg; 3-4; an arm; 5. a head; 6. a torso. If head or torso: save or die. \\
7 & Head bashed. Save or die. Lose $1\die6$: 1. an eye; 2. an ear; 3. a nose; 4. a part of the skull; 5. a jaw; 6. a tongue. \\
8 & Body horror. Save or die. Lose 1 Constitution permanently. \\
9+ & Die. \\
\bottomrule
\end{tabularx}
\end{table}

\footnotetext[1]{Where $H$ stands for Health.
Note that you only roll on this table when your Health is \textit{negative}, so the numbers only go up.}

\footnotetext{To make a save, roll $1\die10$; if it's under your Constitution, you save successfully.}

\section{Magic}\label{magic}

Magic-Users can cast spells.
This takes like two turns to do.
If a Magic-User gets interrupted in casting their spell, well, things get hairy.
The spell automatically fails and I'll roll on a special, super-secret table of horrible things.
Interruption includes being attacked \emph{at all} in melee, or being successfully attacked at range.
However a Magic-User can just stop casting a spell, if they like, with no penalty.

Casting a spell costs mana.
Every Magic-User has mana equal to their level $+1$.
Every morning when the sun rises their mana refills to its maximum.
Magic-Users can spend as much or as little (down to 1) mana as they like on a spell.
Spending more mana will increase the power of the spell.
If a Magic-User spends more mana than they have, they have to make an Intelligence check; failure results in me rolling on my special, super-secret table of horrible things.

\end{document}